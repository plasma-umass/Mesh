Because heap-allocated objects in C and C++ cannot be relocated,
memory allocators for these languages can introduce heap
fragmentation. Existing techniques like segregated-fit allocators
reduce fragmentation compared to alternatives, but they can not
eliminate it, especially under adversarial or worst-case
workloads. Managed languages minimize fragmentation through compacting
garbage collection, where live objects are relocated adjacent to each
other and free space is returned to the operating system.

We introduce meshing: a technique for unmanaged languages that
provides compaction without virtual address relocation. By remapping
virtual pages to point to the same physical page, we mesh objects on
those pages provided their virtual offsets do not overlap. This allows
us to store objects from several virtual pages on a single physical
page, freeing the other physical pages and increasing usable memory.
We present a suite of practical algorithms for memory compaction
through meshing, and develop new theoretical results which guide the
design of these algorithms and probabilistically guarantee their
efficacy.  We implement a meshing memory allocater as a Linux shared
library, libmesh, that works with unmodified C and C++ programs.  We
show that meshing introduces low time overhead in practice, and is
able to reduce memory usage by XX percent over a set of representative
benchmarks.
