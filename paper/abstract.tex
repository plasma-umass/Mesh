Because heap-allocated objects in C and C++ cannot be relocated,
memory allocators for these languages can suffer from
fragmentation. Existing techniques like segregated-fit allocators work
well in practice to reduce fragmentation, but they can not eliminate
it, especially under adversarial or worst-case workloads. Managed
languages minimize fragmentation through compacting garbage
collection, where live objects are relocated adjacent to each other.

We introduce meshing: a technique that can perform compaction without
object relocation for unmanaged languages. By remapping virtual pages
to point to the same physical page, we mesh objects on those pages
provided their virtual offsets do not overlap. This allows us to store
objects from several virtual pages on a single physical page, freeing
the other physical pages and increasing usable memory.  We present a
suite of practical algorithms for memory compaction through meshing,
and develop new theoretical results which guide the design of these
algorithms and probabilistically guarantee their efficacy.  We
implement meshing in libmesh, a memory allocator that works with
unmodified C and C++ programs.  We show that meshing introduces low
time overhead in practice, and is able to reduce memory usage by XX
percent over a set of representative benchmarks.
