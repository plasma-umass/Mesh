\section{Related Work}
\label{sec:related}

BiBoP allocators.

\paragraph{Randomized memory management.} Several previous memory managers have employed randomization,
primarily for fault tolerance and
security~\cite{Novark:2010:DSH:1866307.1866371, 1134000, 1346296,
  1250736}. DieHard uses randomized memory allocation to provide
\emph{probabilistic memory safety}, which enables (probabilistically)
correct execution in the face of memory usage errors like heap buffer
overflows and use-after-free~\cite{1134000}. Archipelago extends
DieHard by placing each object in a randomly-selected page in a 64-bit
address space, significantly extending its ability to tolerate memory
errors~\cite{1346296}. Archipelago reduces its footprint by copying
unused objects into a conventional heap and discards the backing
physical page; it then uses virtual memory page protection to
intercept reads and re-instantiates the physical page on
demand. DieHarder extends DieHard to provide security against attacks
that exploit memory errors~\cite{Novark:2010:DSH:1866307.1866371},
while Exterminator leverages a randomized memory allocator to perform
statistical inference and automatically fix the application by
generating ``runtime patches''~\cite{1250736}.

\paragraph{Exploiting virtual memory.} Numerous memory managers have
exploited virtual memory primitives for a variety of purposes (add
Appel cite, Boehm GC, Archipelago, DieHarder). To our knowledge, the
only one that merges virtual pages is Hound, a memory leak and
``bloat'' detector~\cite{1542521}. To isolate leaks, Hound allocates
objects sequentially and memory protects pages beyond a certain age
(in allocation time) \emph{explain why.} In order to avoid mixing
young and old objects, Hound does not reuse freed slots on pages until
the page becomes completely empty. Because this lack of reuse could
lead to catastrophic memory consumption, Hound performs what we here
call meshing (merging old pages that have become mostly empty). Unlike
Mesh, Hound only considers old pages when performing meshing, and its
deterministic allocation strategy cannot guarantee successful
compaction.
